\documentclass
   [kulak] % options: kul/kulak, handout (altijd kul of kulak optioneel kan handout bijgevoegd worden)
   {kulakbeamer}

\usepackage[dutch]{babel}
\usepackage[utf8]{inputenc}
\usepackage[T1]{fontenc}
\usepackage{listings}
\usepackage{color}
\usepackage{graphicx}
\usepackage{amsmath}
\usepackage{amssymb}
\usepackage{mathrsfs}

\newcommand\tab[1][1cm]{\hspace*{#1}}

\lstdefinestyle{CStyle}{
	breakatwhitespace=false,         
	breaklines=true,                 
	captionpos=b,                    
	keepspaces=true,                 
	numbers=left,                    
	numbersep=5pt,                  
	showspaces=false,                
	showstringspaces=false,
	showtabs=false,                  
	tabsize=2,
	language=C,
	morekeywords={parallel, spawn, sync, new}
}

\title[Simulated annealing]{Job shop scheduling by simulated annealing}
\author[T. Bamelis \& M. Jonckheere]{Thomas Bamelis \& Michiel Jonckheere} 
\institute[Kulak]{KU Leuven Kulak}
\date{Academiejaar 2017-2018}

% Overzicht bij het begin van elk hoofdstuk 
\AtBeginSection[]{\only<beamer>{\addtocounter{framenumber}{-1}
	\begin{outlineframe}[Overzicht]
		\tableofcontents[currentsection]
	\end{outlineframe}}
	}


\defbeamertemplate{description item}{align left}{\insertdescriptionitem\hfill}


\begin{document}

\begin{titleframe}
\titlepage
\end{titleframe}

\begin{outlineframe}[Overzicht]
\tableofcontents
\end{outlineframe}

 % % % Here you go  % % % 

\section{Inleiding}

\begin{frame}
\frametitle{Inleiding}
\textbf{\textit{Simulated annealing}} \\~\\
Een manier om dichtbij optimale oplossing te geven voor een combinatorisch optimalisatie probleem.\\~\\
\pause
Gebaseerd op het afkoelen van metalen. 
\end{frame}

\section[SA algemeen]{De algemene vorm van simulated annealing}

\begin{frame}
\frametitle{Gegevens}
3 benodigdheden voor SA:

\begin{enumerate}
	\item \textbf{$\mathscr{R}$} : Een verzameling configuraties / combinaties
	\item \textbf{$C: \mathscr{R} \rightarrow \mathbb{R}$}: Een functie die de `kost' van een configuratie weergeeft
	\item \textbf{$\mathscr{N}: \mathscr{R} \rightarrow 2^\mathscr{R}$} : Een functie die de `neighborhood' van een configuratie weergeeft, met $\mathscr{N}(i) \subseteq \mathscr{R}$ \\~\\
	Een simpele transitie is nodig waaruit indien toegepast op een configuratie i, alle $\mathscr{N}(i)$ hieruit kunnen volgen
\end{enumerate}
\end{frame}

\begin{frame}
	\frametitle{Algoritme SA}
	$\triangleright$ Iteratief algoritme waarin $i \in \mathscr{R}$ gegeven is bij de initialisatie. \\~\\~\\ \pause
	$\rightarrow$ Er wordt een neighbor $j \in \mathscr{N}(i)$ genomen \\~\\ \pause
	$\rightarrow$ De kans dat we met $j$ verdergaan is min\{$1,e^{\frac{-(C(j)-C(i))}{c}}$\} met $c$ een getal die daalt tijdens de uitvoering (\textit{cfr. temperatuur}).\\~\\
	\pause
	Kans daalt als:
	\begin{itemize}
		\item[$\bullet$] $C(j)-C(i)$ groot of dus j veel zwaarder
		
		\item[$\bullet$] $c$ kleiner wordt of dus tijd vordert
		
		
	\end{itemize}
	\pause
	~\\Kans 1 als $C(j) \leqslant C(i)$
	
	
\end{frame}

\begin{frame}
	\frametitle{c en aantal iteraties}
	Kies voor c begin waarde $\chi\textsubscript{0}$, eindwaard $\epsilon\textsubscript{s}$ en $\delta$
	$c\textsubscript{k+1} = \frac{c\textsubscript{k}}{1 + [c\textsubscript{k}\cdot \ln(1 + \delta)/3\cdot \sigma\textsubscript{k}]}$ \\~\\ met $\sigma\textsubscript{k}$ de standaard afwijking van de kost van de configuraties van de k'de iteratie \\~\\\pause$\rightarrow$ $\delta$ klein gekozen $\Rightarrow$ trage `kwalitatieve' afname c 
	\\~\\\pause
	Lengte ?van de k'de markovketen?:\\
	$L\textsubscript{k} = \displaystyle \max_{i \in \mathscr{R}}\{|\mathscr{N}(i)|\}$
\end{frame}

\begin{frame}
	\frametitle{Kwaliteit algoritme}
	
	$\triangleright$ Indien $\delta$ kleiner dan 0,1 gekozen wordt\\
	is de eindconfiguratie binnen 2\% van het globaal minimum. \\~\\
	Tijdscomplexiteit $= \mathscr{\theta}(\tau L\ln(|\mathscr{R}|))$ met $\tau$ de tijd om een nieuwe transitie door te voeren.
	
\end{frame}

\section[simpeler?]{Waarom niet simpeler}

\begin{frame}
	\frametitle{Waarom niet simpeler?}
	Wat indien men met $j$ verdergaat als $C(j) \leq C(i)$?\\~\\
	\pause
	Stel $i \in \mathscr{R}$ en $\forall j \in \mathscr{N}(i): C(j) \geqslant C(i)$ \\($i$ heeft kleinste kost van al zijn neighbors) \\
	$\Rightarrow$ je zit vast in $\mathscr{N}(j)$ \\~\\
	\pause
	SA `verplicht' om te veranderen in begin en zo niet vast te zitten
\end{frame}

\section[SA job shop]{SA toegepast op job shop scheduling}

\begin{frame}
	\frametitle{Configuraties}
	We geven hier de gegevens die SA nodig heeft door voor ons probleem.\pause 
	~\\~\\  \textbf{\textit{Configuraties}}: \\ ~\\
	De configuratie $i$ wordt weergegeven door $\Pi\textsubscript{i}$ met \\ $\Pi\textsubscript{i} = \{\pi\textsubscript{i1}, ... ,\pi\textsubscript{im}\}$
	met $\pi\textsubscript{ik}$ de volgorde waarin de operaties op machine k worden uitgevoerd. \pause	~\\~\\ $\pi\textsubscript{ik}$ moet $m\textsubscript{k}$ operaties bevatten $\Rightarrow$ $\#\mathscr{R} = \prod\limits_{k=1}^m m\textsubscript{k}!$ met m aantal machines \\~\\ Als operatie v op machine k uitvoert, dan is $\pi\textsubscript{ik}(v)$ de operatie die na v op machine k uitvoert.

\end{frame}

\begin{frame}
	\frametitle{Kost functie}
	%TODO Hier op dezelfde manier als hierboven de 2 gerichte grafen d schrijven en wat ze zijn in mensentaal en dat ze dezelfde lengte hebben
	We definiëren volgende twee grafen:\\
	\begin{enumerate}
		\item $D_i = (V, A \cup E_i)$, met $E_i = \{(v, w) | \{v, w\} \in E$ en 		$\pi_{ik}(v) = w$, voor enkele $k \in \mathscr(M)\}$\\
		\item $\bar{D_i} = (V, A \cup \bar{E_i})$, met $\bar{E_i} = \{(v, w) | \{v, w\} \in E$ en $\pi_{ik}^l(v) = w$, voor enkele $k \in \mathscr(M), 1 \le l \le m_k -1\}$
	\end{enumerate}
	~\\
	
	$\bar{D_i}$ is een disjuncte graaf met bogen uit $E$, waarvan de oriëntatie bepaald wordt door $\Pi_i$.\\ ~\\

	$D_i$ is $\bar{D_i}$ waarvan de bogen uit $\bar{E_i}$ die opeenvolgende operaties zijn van dezelfde machine.
	
	
\end{frame}

\begin{frame}
\frametitle{Kost functie}
	%TODO Zeggen dat de kostfunctie via labelingsalgoritme, normaal is dit het korste pad als alle bogen gewicht -1 krijgen
	Het langste pad van beide grafen zijn even lang.\\~\\
	
	Kost van configuratie $i$ vinden we door het langste pad van $0$ naar $N + 1$ te zoeken in $D_i$ $\rightarrow$ m.b.v. \textit{labeling algoritme}\\~\\
	
	Aantal bogen: $|A| + |E_i| = (N + n) + (N - m)$ \\
	\tab $\Rightarrow$ $O(N)$


\end{frame}

\begin{frame}
\frametitle{Neighborhood Structuur}
	Keuze transitie:\\
	\begin{enumerate}
		\item $v$ en $w$ zijn opeenvolgende operaties op een machine $k$\\
		\item $(v, w)\in E_i$ is een kritieke boog\\ 
	\end{enumerate}
	Daarna $v$ en $w$ wisselen in uitvoering.\\
	\tab $\Rightarrow$ $(u, v)$ en $(w, x)$ wordt $(u, w)$ en $(v, x)$\\~\\
	Enkele transities dus zeker al uitsluiten:\\
	\begin{enumerate}
		\item De kost verlaagt niet na de transitie\\
		\item De transitie resulteert in een cyclic graph\\
	\end{enumerate}
	~\\
	We besluiten dus: $\mathscr{N}(i) < \Sigma_{k=1}^m (m_k - 1) = N - m$
\end{frame}

\begin{frame}
\frametitle{Convergentie}
	%TODO volgens mij zal het toch niet al teveel zeggen als we hier die lemma's en stelling zetten. Lemma 2 kunnen we wel eens aanhalen in vorige slide.
\end{frame}

%TODO Over section 4 moeten we ook niet echt iets zeggen denk ik? ofwel eens aanhalen dat SA toch wel beter werkt als het gaat om grote problemen, en die andere algoritmes eens aanhalen?

\section{Besluit}
\begin{frame}
\frametitle{Besluit}
SA blijft relatief simpel en vermijd de valkuil van de simpelere methode.
\end{frame}

\end{document}
